%\VignetteIndexEntry{LPEseq Vignette}
%\VignetteKeyword{RNA-Seq}
%\VignetteKeyword{differential expression}
%\VignettePackage{LPEseq}
\documentclass[12pt]{article}

\textwidth=6.2in
\textheight=8.5in
\oddsidemargin=0.2in
\evensidemargin=0.2in
\headheight=0in
\headsep=0in

\usepackage{Sweave}
\begin{document}
\input{LPEseq-concordance}

\title{LPEseq Package Introduction}
\author{Jungsoo Gim and Taesung Park}
\date{26 Aug 2015}
\maketitle


\section{Introduction}
The LPEseq is an R package for performing differential expression (DE) test with
RNA sequencing data. Briefly, LPEseq extends local pooled error method, which
was developed for microarray data analysis, to sequencing data even with
non-replicated sample in each condition. A number of methods are available for
both count-based and FPKM-based RNA-Seq data. Among these methods, few (for
example, EdgeR and DESeq) can deal with no replicate data, but not accurately.
LPEseq was designed for the RNA-Seq data with a small number of replicates,
especially with non-replicate in each class. Also LPEseq can be equally applied
both count-base and FPKM-based (non-count values) input data. This brief
vignette is written for the users who want to use the LPEseq for their DE
analysis. An extended documentation about the method can be found in our
original manuscript \texttt{(http://bibs.snu.ac.kr/software/LPEseq).}

The full LPEseq User's Guide (with figures) is available in our website.
Please visit\texttt{http://bibs.snu.ac.kr/software/LPEseq} and a pdf file is available.


\section{Installation}
The source code and a package of LPEseq are freely available from our website
and from Bionconductor. You can use it by loading source code in our website.

\begin{Schunk}
\begin{Sinput}
> install.packages("LPEseq")